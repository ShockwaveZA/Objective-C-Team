\documentclass[12pt,a4paper]{article}
\usepackage[margin=2.5cm,left=2cm,includefoot]{geometry}
\usepackage{hyperref}
\usepackage{array}
\usepackage{enumitem}
\usepackage{graphicx}
\usepackage[section]{placeins}
\usepackage{titlesec}

\setlength{\parindent}{0em}

\graphicspath{{Images/}}

% Header and footer
\usepackage{fancyhdr}
\pagestyle{fancy}

\rhead{COS 301}
\lhead{Team Objective C}
\fancyfoot{}
\fancyfoot[R]{Page \thepage}

\renewcommand{\headrulewidth}{2pt}
\renewcommand{\footrulewidth}{1pt}

\begin{document}

\begin{titlepage}
  \begin{center}
    \begin{figure}[t]
      \centering
      \includegraphics[width=350px]{logo.PNG}
     \end{figure}
     
     \textsc{\LARGE COS301 Group Task 2 \newline \newline Architectural Design\\[0.5cm] Specifications}
     
     \textbf{\newline Team Objective C} \\
     \begin{flushright} \large
      Diana Obo \emph{u13134885}\newline
      Kamogelo Tsipa \emph{u13010931}\newline
      Linda Zwane \emph{u14199468}\newline
      Melvin Zitha \emph{u12138747}\newline
      Minal Pramlall \emph{u13288157}\newline
      Rotondwa Siavhe \emph{u????????}\newline
       \end{flushright} 
      \vfill %whitespace
      
      Team Objective C Github: \href{https://github.com/ShockwaveZA/Objective-C-Team}{Github} page.\\
      \url{https://github.com/ShockwaveZA/Objective-C-Team}
      
      \vfill
      {\large Date:}
      \\
      {\large \today}
     \end{center}
    \end{titlepage}


\tableofcontents
\newpage

\section{Quality and feasibility of design}
Given the scope of the NavUP system, we must ensure that the operational requirements of the system, and their incorporation into design requirements, can be converted into a finished product. Given the design constrains, see Design constraints, the NavUP system will reliably provide users with basic operational requirements such as navigation functionalities and access to a wealth of information on campus about events, venues and points of interests based on user input. The design checks and validates input from the user to ensure that the results returned are correct and complete, in a consistent manner. In cases where the user has entered anything incorrectly, the robust design of the system will enable it to still behave as expected or stop and request the user to enter the details again. \newline

When it comes to efficiency, the NavUP system operates on any device that matches the minimum requirements, see minimum requirements. WIFI, cellular or Global positioning system (GPS) network will be used to communicate efficiently with the server. The system will also be able to perform under abnormal conditions, such as when the WIFI network is not available, cellular network will be used to allow the user to use the application. \newline

To make sure that maintainability of the system is easy, it is divided into different modules. The modules contain within them, subsystems which allow for additional features to be added or removed. The systems modular design allows it to continue operating if other modules fail to respond. \newline

The general usability of the application allows users of all levels the ability to use the application. As with the user interface, see External Interface Requirements, it is designed to allow users access to basic features of the application. User input validation will be used to make sure that users enter the correct information and the correct information is sent back to the user. \newline


\end{document}
